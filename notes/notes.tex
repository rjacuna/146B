\documentclass{article}
\usepackage{fontspec}
\usepackage{xcolor}
%\usepackage{sagetex}

\usepackage{euler}
\usepackage{amsmath}
\usepackage{amssymb}
\usepackage{unicode-math}


\usepackage[makeroom]{cancel}
\usepackage{ulem}

\setlength\parindent{0em}
\setlength\parskip{0.618em}
\usepackage[a4paper,lmargin=1in,rmargin=1in,tmargin=1in,bmargin=1in]{geometry}

\setmainfont[Mapping=tex-text]{Helvetica Neue LT Std 45 Light}

\newcommand\N{\mathbb{N}}
\newcommand\Z{\mathbb{Z}}
\newcommand\Q{\mathbb{Q}}
\newcommand\R{\mathbb{R}}
\newcommand\C{\mathbb{C}}
\newcommand\A{\mathbb{A}}

\usepackage{soul}
\begin{document}

\begin{center}
  146B --- Fourier Series
\end{center}\vspace{1.618em}

Let $f:[-L,L]\rightarrow \R$.

$$f \sim \frac{a_0}{2} + \sum_{n=1}^\infty [a_n \cos(\frac{n\pi x}{L}) +
b_n sin(\frac{n\pi x}{L})]$$

$$a_n = \frac{1}{L} \int_{-L}^L f(x)\cos(\frac{n\pi x}{L}) dx\enskip,
n\in \N_0$$

$$b_n = \frac{1}{L} \int_{-L}^L f(x)\sin(\frac{n\pi x}{L}) dx\enkskip,
n\in \N$$

$$1^\circ f\text{ is odd: } a_n = 0; 2^\circ f\text{ is even: } b_n = 0$$

Convergence

$f$ is said to be piecewise continuous on $[a,b]$, if the interval is
divided into a finite number of subintervals,
$(x_0,x_1),(x_1,x_2),\cdots,(x_{n-1},x_n)$, $a = x_0$, $b = x_n$. So
that,

$$1^\circ f\text{ is continuous on} (x_{i-1}, x_i),\text{ for } i =
1,2,\dots,n$$
$$2^\circ f\text{ has finite (one-sided) limit at the endpoints }
x_i,\text{ for } i = 1,2,\dots, n$$

Ass. $f$ and its derivative $f^\prime$ are piecewise continuous on
$[-L,L].$ $f$ has a period of $2L$. Then $f$ has a Fourier expansion
$$f \sim \frac{a_0}{2} + \sum_{n=1}^\infty [a_n \cos(\frac{n\pi x}{L}) +
b_n \sin(\frac{n\pi x}{L})]$$
The Fourier series converges to $f(x)$ where $f$ is continuous, and
converges to,$$\frac{f(x+)+f(x-)}{2}$$ at points where $f$ is
discontinuous. The value of $f$ at the discontinuities need not be the
average of the left and right hand limits.

The heat equation $$\begin{cases} u_t = u_{xx}, 0<x<L\text{ and }t>0\\
  u(0,t)= u(L,t) = 0 \\ u(x,0) = f(x)\end{cases}$$

Separation of variables, for the eigenvalues $$\lambda_n =
(\frac{n\pi}{L})^2,$$ with corresponding
eigenfunctions $$X_n(x) = \sin(\frac{n\pi}{L}x),\text{ and }T_n(x) =
e^{-(\frac{n\pi}{L})^2 t}$$.

And $u_n(x,t)= X_n(x)T_n(x) =
sin(\frac{n\pi}{L}x)e^{-(\frac{n\pi}{L})^2 t}$ for $n\in \N$.

Then the general solution is $$u(x,t) = \sum_{n=1}^\infty C_n
\sin(\frac{n\pi}{L}x)e^{-(\frac{n\pi}{L})^2 t}$$
\newpage

We can see that the last condition gives us that,
$$u(x,0) = \sum_{n=1}^\infty C_n
\sin(\frac{n\pi}{L}x)e^{-(\frac{n\pi}{L})^2 0} = \sum_{n=1}^\infty c_n
\sin(\frac{n\pi}{L}x) = f(x)$$

We need to extend $f:[0,L]\rightarrow \R$
$[-L,L]$ to a function $F:[-L,L]\rightarrow \R$ to get a Fourier expansion of $F$.

And we need the extension to be an odd function to get a Fourier expansion involving only
$\sin(\frac{n\pi}{L}x)$.

Then define $F:[-L,L]\rightarrow \R; F(x) = \begin{cases} f(x)\quad\enskip \text{, for
  }\quad\enskip 0<x<L\\ -f(-x)\text{, for }-L<x<0\end{cases}$

$$b_n = \frac{1}{L}\int_{-L}^L F(x)\sin(\frac{n\pi}{L}x)dx =
\frac{2}{L}\int_{0}^LF(x)\sin(\frac{n\pi}{L}x)dx\enskip,\quad \text{ since
}F(x)\sin(\frac{n\pi}{L}x)\text{ is even.}$$

Notice $F = f$ on $[0,L]$ and $c_n$ corresponds to $b_n$ so,

$$c_n = \frac{2}{L} \int_0^L f(x) \sin(\frac{n\pi}{L}x) dx.$$


Now to consider general classes of solutions of partial differential
equations, we need to see if the first two conditions determine a
series of $\cos(\frac{n\pi}{L}x)$. So, we need to consider the even
extension $F$ of $f$,

Define $F:[-L,L]\rightarrow \R; F(x) = \begin{cases} f(x)\quad\enskip \text{, for
  }\quad\enskip 0<x<L\\f(-x)\enskip\text{ , for }-L<x<0\end{cases}$

Similarly, $a_n = \frac{2}{L} \int_{0}^L f(x)\cos(\frac{n\pi}{L}x)dx.$
This is called the cosine expansion, and the $c_n$ are called the sine
expansion. Also called the half-range expansions for
$f:[0,L]\rightarrow \R$, corresponding to sine and cosine series.


\end{document}


%%% Local Variables:
%%% mode: latex
%%% TeX-master: t
%%% End:
