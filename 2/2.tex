\documentclass{article}
\usepackage{fontspec}
\usepackage{xcolor}
\usepackage{sagetex}

\usepackage{euler}
\usepackage{amsmath}
\usepackage{amssymb}
\usepackage{unicode-math}


\usepackage[makeroom]{cancel}
\usepackage{ulem}

\setlength\parindent{0em}
\setlength\parskip{0.618em}
\usepackage[a4paper,lmargin=1in,rmargin=1in,tmargin=1in,bmargin=1in]{geometry}

\setmainfont[Mapping=tex-text]{Helvetica Neue LT Std 45 Light}

\newcommand\N{\mathbb{N}}
\newcommand\Z{\mathbb{Z}}
\newcommand\R{\mathbb{R}}
\newcommand\C{\mathbb{C}}
\newcommand\A{\mathbb{A}}

\begin{document}

\begin{center}
  146B --- 2

  RJ Acuna

  (862079740)
\end{center}\vspace{1.618em}

\subsection*{Section 3.6}

In each of Problems $1$ through $4$, use the method of variation of parameters to find a particular
solution of the given differential equation. Then check your answer by using the method of
undetermined coefficients.

\paragraph{3} $y^{\prime\prime} + 2y^\prime + y = 3e^{-t}$

\uwave{slu.}

$r^2+2r+1 = 0 \implies r = \frac{-2\pm\sqrt{4-4}}{2} = -1$

$\implies \{e^{-t},te^{-t}\}$ is a fundamental set of solutions.

$W[e^{-t},te^{-t}] = \begin{vmatrix}e^{-t}&te^{-t}\\ -e^{-t}&
  (1-t)e^{-t}\end{vmatrix} = (1-t)e^{-2t}-(-te^{-2t}) = e^{-2t}$

$W_1[e^{-t},te^{-t}] = \begin{vmatrix}0&te^{-t}\\ 1&
  (1-t)e^{-t}\end{vmatrix} = -te^{-t}$

$W_2[e^{-t},te^{-t}] = \begin{vmatrix}e^{-t}&0\\ -e^{-t}&
  1\end{vmatrix} = e^{-t}$

So, by Alysha's notes:

$u_1(t) = \int \frac{3e^{-t}(-te^{-t})}{e^{-2t}} dt =-3 \int
\frac{te^{-2t}}{e^{-2t}} dt = -3\int t dt$


$u_1(t) = -\frac{3}{2}t^2 + C$

$u_2(t) = \int \frac{3e^{-t}(e^{-t})}{e^{-2t}} dt = 3 \int 1 dt =
3t + C$

The $C$s give the homogeneous solution so they don't add anything.
\begin{align*}
y_p(t) &= (-\frac{3}{2}t^2)e^{-t}
      +(3t)te^{-t}\\
    &= \frac{3}{2}t^2e^{-t}
\end{align*}

The method of undetermined coefficients yields,

$Y = At^2e^{-t}$$\implies Y^\prime = A(2te^{-t}-t^2e^{-t}) \implies
Y^{\prime\prime} = A(2e^{-t}-4te^{-t} +t^2e^{-t})$

$\implies L[Y] = 2Ae^{-t} = 3e^{-t} \implies A=\frac{3}{2} \implies
y_p(t) = \frac{3}{2}t^2e^{-t} \quad \lozenge$
\newpage
In each of Problems $5$ through $12$, find the general solution of the
given differential equation.

\paragraph{5} $y^{\prime} + y = \tan (t)\enskip,\enskip 0 < t <
π/2\enskip.$
\begin{sagesilent}
  from sage.symbolic.integration.integral import indefinite_integral
  def simp(f):
      return f.simplify_full().simplify_trig()
  t = var('t')
  y1 = cos(t)
  y2 = sin(t)
  g = tan(t)
  w = simp(wronskian(y1,y2))
  u1 = simp(indefinite_integral(tan(t)*-1*y2/w,t))
  u2 = simp(indefinite_integral(tan(t)*y1/w,t))
  yp = simp(y1*u1 + y2*u2)
\end{sagesilent}
\uwave{slu.}

$r^2+1=0 \implies r=\pm i \implies \phi:=\{\cos(t),\sin(t)\}$ is a
fundamental set of solutions. $W[\phi](t) = \sage{w}$.

Then it follows that,
\begin{align*}
u_1(t) &= \int \tan(t)(-\sin(t)) dt\\
        &= \sage{u1}
\end{align*}
\begin{align*}
 u_2(t)&= \int \tan(t)\cos(t) dt\\
       &= \sage{u2}
\end{align*}
\begin{align*}
  y_p(t) &= \cos(t)(\sage{u1}) +\sin(t)(\sage{u2})\\
         &= \sage{y1*u1} +\sage{y2*u2}\\
         &= \sage{yp}
\end{align*}

$y(t) = A\cos(t) + B\sin(t) \sage{yp} \quad \blacklozenge$

\paragraph{7} $y^{\prime\prime} +4y^\prime +4y = t^{-2}e^{-2t}.$

\begin{sagesilent}
  from sage.symbolic.integration.integral import indefinite_integral
  def simp(f):
      return f.simplify_full().simplify_trig()
  t = var('t')
  y1 = e^(-2*t)
  y2 = t*e^(-2*t)
  g = t^(-2)*e^(-2*t)
  w = wronskian(y1,y2)
  sw = simp(w)
  u1 = indefinite_integral(g*-1*y2/w,t)
  u2 = indefinite_integral(g*y1/w,t)
  yp = y1*u1 + y2*u2
\end{sagesilent}

\uwave{slu.}

$r^2+4r^2+4 = 0 \implies r = \frac{-4\pm\sqrt{16-16}}{2} = -2 \implies
\phi := \{e^{-2t},te^{-2t}\}$ is a fundamental set of solutions for
$7$.

$W[\phi](t) = \sage{w} =\sage{sw}$

\begin{align*}
  u_1(t) &= \int \frac{\sage{g}(-\sage{y2})}{\sage{sw}} dt\\
         &= \sage{u1} + C
\end{align*}
\begin{align*}
  u_2(t) &= \int \frac{\sage{g}(\sage{y1})}{\sage{sw}} dt\\
         &= \sage{u2} + C
\end{align*}

Then,

\begin{align*}
  y_p(t) &= \sage{y1}\sage{u1}+\sage{y2}\sage{u2}\\
         &= \sage{yp}
\end{align*}

Since, $-e^{-t}$ is in $\phi$, it follows it gets absorbed by the
constant in the general solution.

\[y(t) = A\sage{y1}+B\sage{y2} \sage{yp + y1}\quad \blacklozenge \]

\subsection*{Section 5.1}

In each of the Problems $1$ through $8$, determine the radius of
convergence of the given power series.

\paragraph{5} $\sum_{n=1}^{\infty} \frac{(2x+1)^n}{n^2}$

\uwave{slu.}
$\frac{1}{\rho} =\lim_{n\rightarrow \infty}
\left|\frac{n^2}{(n+1)^2}\right| = \lim_{n\rightarrow \infty}
\left|\frac{n^2}{n^2+2n+1}\right| = 1$

$\implies \rho = 1\quad\lozenge$

\paragraph{7} $\sum_{n=1}^{\infty} \frac{(-1)^n n^2(x+2)^n}{3^n}$

\uwave{slu.}
$\frac{1}{\rho} = \lim_{n\rightarrow \infty}
\sqrt[n]{\left|\frac{(-1)^n n^2}{3^n}\right|} =\lim_{n\rightarrow \infty}
\frac{n^{\frac{2}{n}}}{3} = \frac{1}{3}$

$\implies \rho = 3 \quad \lozenge$

In each of Problems $9$ through $16$, determine the Taylor series
about the point $x_0$ for the given function. Also determine the
radius of convergence of the series.

\paragraph{15} $\frac{1}{1-x}\quad,\quad x_0 = 0$

\uwave{slu.}

$\frac{1}{1-x} = \sum_{n=0}^{\infty} x^n$, by taking the
limit of the geometric series as $n\rightarrow \infty$.

$\rho = 1$, because otherwise the series diverges$\quad \lozenge$

\paragraph{18} Given that $y = \sum_{n=0}^{\infty} a_nx^n$ compute
$y^\prime$ and $y^{\prime\prime}$ and write out the first four terms of each
series, as well as the coefficient of $x^n$ in the general term. Show that if $y^{\prime\prime} = y$, then the
coefficients $a_0$ and $a_1$ are arbitrary, and determine $a_2$ and $a_3$ in terms of $a_0$ and $a_1$. Show
that $a_{n+2} = \frac{a_n}{(n + 2)(n + 1)}, n = 0, 1, 2, 3, \dots$

\uwave{pf.}

$\quad\quad\enskip y = \sum_{n=0}^{\infty} a_nx^n  = a_0+a_1x+a_2x^2+a_3x^3+ a_4x^4+a_5x^5+\dots$

$\implies y^{\prime} = \sum_{n=1}^{\infty} na_nx^{n-1} = a_1 +2a_2x +3a_3x^2+4a_4x^3+5a_5x^4+\dots$

$\implies y^{\prime\prime} = \sum_{n=2}^{\infty} n(n-1)a_nx^{n-2} = 2a_2+6a_3x+12a_4x^2+20a_5x^3+\dots$

Let $k = n-1 \implies n = k+1\quad \implies y^{\prime} =
\sum_{k=0}^{\infty} (k+1)a_{k+1}x^{k}$

Let $s = n-2 \implies n = s+2\quad \implies y^{\prime\prime} = \sum_{s=0}^{\infty} (s+2)(s+1)a_{s+2}x^{s}$

Let $s = n$, and ass. $y^{\prime\prime} = y$, then,

Looking at the computed coefficients we can see,

$a_0 = 2a_2 \implies a_2 = \frac{a_0}{2}$
and
$a_1 = 6a_3 \implies a_3 = \frac{a_1}{6}$
Then notice that in the general terms,
\[\sum_{n=0}^{\infty} a_nx^n = \sum_{n=0}^{\infty}
  (n+2)(n+1)a_{n+2}x^{n}\]
\[\implies a_n = (n+2)(n+1)a_{n+2}\]
\[\implies a_{n+2} = \frac{a_n}{(n+2)(n+1)}\]

So, at a glance all the even terms depend on $a_0$, and the odd terms
on $a_1\quad \blacksquare$

\paragraph{28} Determine $a_n$ so that the equation
\[\sum_{n=1}^{\infty} na_nx^{n-1} +2 \sum_{n=0}^{\infty} a_nx^n = 0\]
is satisfied. Try to identify the function represented by
$\sum_{n=0}^{\infty} a_nx^n\enskip.$

\uwave{pf.}

\begin{align*}
  \sum_{n=1}^{\infty} na_nx^{n-1} &+2 \sum_{n=0}^{\infty} a_nx^n = 0\\
  \sum_{n=1}^{\infty} na_nx^{n-1} &= -2 \sum_{n=0}^{\infty} a_nx^n\\
  \sum_{n=0}^{\infty} (n+1)a_{n+1}x^{n} &= -2 \sum_{n=0}^{\infty}
                                          a_nx^n\\
  \implies (n+1)a_{n+1} &= -2a_n\\
  \implies a_{n+1} &= -\frac{2a_n}{n+1}
\end{align*}

\begin{align*}
  n=0 &\quad a_1 =-\frac{2a_0}{1} = -\frac{2a_0}{1}\\
  n=1 &\quad a_{2} = -\frac{2a_1}{2} = -\frac{2a_1}{2} =
        -\frac{2(-2a_0)}{2\cdot 1}= (-1)^{2}\frac{2^2a_0}{2\cdot1}\\
  n=2 &\quad a_{3} = -\frac{2a_2}{3} =
        (-1)^3\frac{2^3a_0}{3\cdot 2\cdot 1}\\
  &\vdots\\
  n=k-1 &\quad a_{k+1} = \frac{(-1)^{k+1}2^{k+1}a_0}{k!} =
        -2a_0\frac{(-1)^{k}2^{k}}{k!}
\end{align*}

So, the function is $-2a_0e^{-2x}\quad$
$\blacksquare$


\subsection*{Section 5.2}

\paragraph{1} $y^{\prime\prime} − y = 0, x_0 = 0$

\uwave{slu.}

We can see that $a_{n+2} = \frac{a_n}{(n+2)(n+1)},\enskip$ from
problem \textbf{18}.

\begin{align*}
  n=0 &\quad a_2 =\frac{a_0}{2} &\quad n=1 &\quad a_{3} = \frac{a_1}{6} \\
  n=2 &\quad a_{4} = \frac{a_2}{(4)(3)} = \frac{a_0}{4!}  &\quad n=3
                                           &\quad a_{5} =
                                             \frac{a_3}{(5)(4)} =
                                             \frac{a_0}{5!}\\
      &\vdots &\quad &\vdots\\
n=2k &\quad a_{2k} = \frac{a_0}{(2k)!}  &\quad n=2k+1   &\quad a_{2k+1} = \frac{a_1}{(2k+1)!}
\end{align*}

So, $\cdots$ I don't know I didn't have time to finish... Probably $
e^{t},$ and $e^{-t}.$

%\newpage
%\paragraph{2}

%\paragraph{10}

%\subsection*{Section 5.3}

%\paragraph{6}

%\subsection*{Section 5.4}

%\paragraph{1}

%\paragraph{3}

%\paragraph{4}

%\paragraph{6}

%\paragraph{17}

%\paragraph{18}

%\paragraph{19}

%\paragraph{22}

%\subsection*{Section 5.5}

%\paragraph{1}

%\paragraph{3}


\end{document}


%%% Local Variables:
%%% mode: latex
%%% TeX-master: t
%%% End:
